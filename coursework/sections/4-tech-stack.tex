\section{Выбор инструментальных средств разработки}

В процессе разработки распределенной системы для обработки финансовых данных в реальном времени я использовал ряд инструментальных средств.

Основным языком программирования, который я использовал, был Java 17. Это популярный язык программирования, который позволяет разрабатывать высокопроизводительные и масштабируемые приложения.

Для разработки микросервисов я использовал Spring Boot и Spring Cloud OpenFeign. Spring Boot - это фреймворк для разработки микросервисов на языке Java, который позволяет ускорить процесс разработки и упростить конфигурацию приложения. Spring Cloud OpenFeign - это библиотека, которая позволяет создавать клиенты для микросервисов на основе аннотаций.

Для обмена сообщениями между микросервисами я использовал Apache Kafka. Это распределенная платформа для потоковой обработки данных, которая позволяет обрабатывать потоки данных в реальном времени.

Для контейнеризации приложения я использовал Docker и docker-compose. Docker - это платформа для создания, развертывания и управления контейнерами, которая позволяет упростить процесс разработки и развертывания приложения. Docker-compose - это инструмент, который позволяет запускать несколько контейнеров вместе и управлять ими.

Для мониторинга и управления приложением я использовал Prometheus и Grafana. Prometheus - это система мониторинга, которая позволяет собирать метрики приложения и анализировать их. Grafana - это инструмент визуализации данных, который позволяет отображать метрики приложения в виде графиков и диаграмм.

Для хранения данных я использовал базу данных PostgreSQL и Spring Data JPA. PostgreSQL - это реляционная база данных, которая позволяет хранить и управлять данными. Spring Data JPA - это библиотека, которая позволяет упростить работу с базой данных и ускорить процесс разработки.

Каждый из этих инструментов был выбран с учетом требований проекта и доступных ресурсов. Вместе они позволили разработать высокопроизводительную и масштабируемую систему для обработки финансовых данных в реальном времени.