\section{Выбор инструментальных средств разработки}

Для реализации нашего проекта был выбран ряд инструментальных средств и технологий, обеспечивающих высокую производительность, масштабируемость и удобство разработки. Мы использовали микросервисную архитектуру с паттерном \textbf{shared database} и применили \textbf{gRPC} для межсервисного взаимодействия.

\textbf{Микросервисная архитектура:}
Данный подход позволяет разделить приложение на независимые компоненты (микросервисы), каждый из которых отвечает за выполнение определенной функциональности. Это повышает гибкость разработки и упрощает масштабирование системы.

\textbf{Языки программирования и фреймворки:}
\begin{itemize}
    \item \textbf{Go:} Часть микросервисов была написана на языке Go, что обеспечило высокую производительность и низкое потребление ресурсов. Разработкой этих микросервисов занимался Максим.
    \item \textbf{Java с использованием Spring Boot 3:} Другие микросервисы были реализованы на Java с использованием фреймворка Spring Boot 3. Этот фреймворк предоставляет широкие возможности для создания производительных и надежных приложений.
\end{itemize}

\textbf{Базы данных:}
\begin{itemize}
    \item \textbf{PostgreSQL:} Основная база данных нашего приложения. PostgreSQL был выбран за его надежность, соответствие стандартам SQL и богатый функционал.
    \item \textbf{ScyllaDB:} Использовалась для реализации мессенджера благодаря своей высокой производительности и способности обрабатывать большие объемы данных с минимальной задержкой.
\end{itemize}

\textbf{Мониторинг и наблюдаемость:}
Для обеспечения мониторинга и наблюдаемости (observability) системы мы использовали следующие инструменты:
\begin{itemize}
    \item \textbf{Prometheus:} Система мониторинга и сбора метрик, которая позволяет отслеживать состояние микросервисов и базы данных.
    \item \textbf{Grafana:} Платформа для визуализации данных, интегрированная с Prometheus, которая предоставляет удобный интерфейс для анализа метрик и мониторинга состояния системы.
    \item \textbf{Loki:} Система для управления и анализа логов, что позволяет эффективно отслеживать и устранять возникающие в приложении проблемы.
\end{itemize}

\textbf{gRPC:}
Для межсервисного взаимодействия мы использовали gRPC, который обеспечивает высокую производительность и низкие задержки при обмене данными между микросервисами. Это позволило нам создать эффективную и масштабируемую систему.

\textbf{CI/CD:}
Для автоматизации процессов сборки, тестирования и развертывания мы использовали следующие инструменты:
\begin{itemize}
    \item \textbf{Github Actions} Система непрерывной интеграции, которая автоматизирует процессы сборки и тестирования кода.
    \item \textbf{Docker:} Платформа для контейнеризации приложений, что обеспечивает удобство развертывания и масштабирования системы.
    \item \textbf{Kubernetes:} Оркестратор контейнеров, который позволяет управлять и масштабировать контейнеры в распределенной среде.
    \item \textbf{Helm:} Пакетный менеджер для Kubernetes, который упрощает управление конфигурациями и релизами приложений.
\end{itemize}