\section{Методика и результаты тестирования}

Для тестирования программной системы я использовал различные методы и инструменты. Я провел ручное тестирование, а также интеграционное тестирование с помощью Spring Boot Test и testcontainers, а также unit тестирование с помощью JUnit5 и Mockito.

Ручное тестирование позволило проверить правильность работы системы и выявить возможные ошибки и проблемы. Также было произведено тестирование на различных этапах разработки, начиная с тестирования каждого отдельного сервиса и заканчивая тестированием всей системы в целом.

Интеграционное тестирование с помощью Spring Boot Test и testcontainers позволило проверить работу системы в целом, а также проверить работу каждого сервиса внутри системы. Было использовано testcontainers для запуска контейнеров с базой данных PostgreSQL и кафкой, что позволило проводить тестирование в реальных условиях.

Unit тестирование с помощью JUnit5 и Mockito позволило проверить правильность работы каждого отдельного компонента системы. Было использовано Mockito для подмены зависимостей и создания тестовых сценариев.

В целом, методика тестирования была основана на принципах современной методологии тестирования, которая включает
в себя ряд принципов и методов,
которые помогают создавать качественные, надежные и эффективные программные
продукты. Вот некоторые из них:

\begin{itemize}
    \item Автоматизированное тестирование. Автоматизированные тесты позволяют повторять тестовые сценарии быстро и точно, что позволяет выявлять проблемы и ошибки на ранних стадиях разработки.
    \item Интеграционное тестирование. Интеграционное тестирование позволяет проверять, как различные компоненты программного продукта взаимодействуют друг с другом, что позволяет выявлять проблемы и ошибки на ранних стадиях разработки.
    \item Unit-тестирование. Unit-тестирование позволяет проверять отдельные модули кода на корректность и соответствие требованиям, что позволяет выявлять проблемы и ошибки на ранних стадиях разработки.
    \item Контейнеризация для интеграционного тестирования. Контейнеризация позволяет создавать изолированные среды для интеграционного тестирования, что помогает избежать проблем с зависимостями и конфигурацией.
    \item Мокирование зависимостей для unit-тестирования. Мокирование зависимостей позволяет создавать тестовые сценарии для отдельных модулей кода, не зависящих от других модулей, что позволяет выявлять проблемы и ошибки на ранних стадиях разработки.
\end{itemize}

Эти принципы и методы помогают создавать качественные программные продукты и
обеспечивать их надежность и эффективность.

После проведения тестирования получились положительные результаты. Система работает стабильно и быстро обрабатывает финансовые данные в реальном времени. Были выявлены и устранены некоторые ошибки и проблемы в процессе тестирования, что позволило создать надежную и эффективную систему для обработки финансовых данных. В результате работы получилась высокоэффективная система, которая может быть использована в различных областях финансовой деятельности.