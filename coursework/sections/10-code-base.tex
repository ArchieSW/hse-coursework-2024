\section{Ссылка на хранилище исходного кода}

Весь исходный код программного обеспечения находится в организации на \href{
      https://github.com/orgs/HSECourseWork2022/repositories}{Github}.

Перечислю и опишу используемые репозитории:
\begin{enumerate}
      \item Репозиторий \href{https://github.com/HSECourseWork2022/dev}{dev}
            хранит файл docker-compose для запуска всего приложения, а также файлы
            описывающие дашборд для Grafana.

      \item Репозиторий \href{https://github.com/HSECourseWork2022/cmc-producer-ms}{cmc-producer-ms}
            хранит исходных код сервиса для поставления данных с провайдера данных CoinMarketCap,
            использующийся для поставления данных о криптовалютах.

      \item Репозиторий \href{https://github.com/HSECourseWork2022/cmc-filter-mus}{cmc-filter-ms}
            хранит исходных код для сервиса, который приводит вид из формата, поставляемого
            cmc-producer-ms в тип сущности, готовую в отгрузке в базу данных.

      \item Репозиторий \href{https://github.com/HSECourseWork2022/cryptocurrency-loader-ms}
            {cryptocurrency-loader-ms}
            хранит исходный код сервиса для загрузки отфильтрованных данных в базу данных.

      \item Репозиторий \href{https://github.com/HSECourseWork2022/report-builder}{report-builder}
            хранит исходный код сервиса для формирования данных для конечного пользователя.

      \item Репозиторий \href{https://github.com/HSECourseWork2022/db-migrations}{db-migrations}
            хранит файл docker образа, для миграции в базу данных с использованием Liquibase.

      \item Репозиторий \href{https://github.com/HSECourseWork2022/models-specs}{models-specs}
            хранит файлы для спецификаций сервисов, а также OpenAPI документация для сервиса report-builder

\end{enumerate}