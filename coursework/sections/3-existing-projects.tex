\section{Обзор существующих решений}

Существует несколько систем для обработки финансовых данных в реальном времени.
Одной из таких систем является Bloomberg Terminal. Это финансовая платформа,
которая предоставляет доступ к рыночным данным, новостям, аналитическим
инструментам и торговым системам. Bloomberg Terminal является популярным
инструментом для трейдеров и инвесторов, так как позволяет получать актуальную
информацию о финансовых рынках в режиме реального времени.

Еще одной системой для обработки финансовых данных в реальном времени является
Thomson Reuters Eikon. Это платформа, которая предоставляет доступ к рыночным
данным, новостям, аналитическим инструментам и торговым системам. Thomson
Reuters Eikon также позволяет получать актуальную информацию о финансовых рынках
в режиме реального времени.

Кроме того, существуют системы, которые предоставляют доступ к финансовым
данным в режиме реального времени через API. Например, Alpha Vantage API
предоставляет доступ к актуальным данным о финансовых рынках, таким как
котировки, графики и новости. Alpha Vantage API имеет бесплатный план, который
позволяет получать до 500 запросов в день, а также платные планы с более
высокими лимитами.

Каждая из этих систем имеет свои преимущества и недостатки, и выбор конкретной
системы зависит от требований проекта и доступных ресурсов. В разработке
распределенной системы для обработки финансовых данных в реальном времени
необходимо учитывать не только функциональные требования, но и требования по
производительности, масштабируемости и отказоустойчивости.