\section{Пути возможного развития}

На основе результатов проведенного нагрузочного тестирования и анализа текущих возможностей системы, были выявлены несколько направлений для дальнейшего развития и улучшения нашего приложения для знакомств.

\textbf{Улучшение архитектуры мессенджера:}
Результаты нагрузочного тестирования показали, что архитектура мессенджера требует оптимизации для обеспечения стабильной работы под высокими нагрузками. Для этого планируется:
\begin{itemize}
    \item Оптимизировать текущую архитектуру мессенджера для повышения производительности.
    \item Внедрить механизмы горизонтального масштабирования для обработки большого количества одновременных пользователей.
    \item Улучшить механизмы кэширования и обработки сообщений для снижения задержек.
\end{itemize}

\textbf{Расширение функциональности мессенджера:}
Для повышения удобства и расширения возможностей взаимодействия между пользователями, необходимо добавить следующие функции:
\begin{itemize}
    \item Возможность отправки голосовых сообщений, что позволит пользователям выражать свои эмоции и мысли более полно.
    \item Поддержка отправки сообщений с фотографиями и файлами для разнообразия общения и обмена информацией.
\end{itemize}

\textbf{Разработка рекомендательной системы:}
Для улучшения качества знакомств и повышения удовлетворенности пользователей необходимо внедрить продвинутую рекомендательную систему. Эта система будет:
\begin{itemize}
    \item Основываться на психологическом портрете пользователя, учитывая его интересы, предпочтения и поведение в приложении.
    \item Использовать машинное обучение для персонализации рекомендаций и повышения их точности.
    \item Интегрироваться с текущими микросервисами для обеспечения бесшовного взаимодействия и актуальности данных.
\end{itemize}

\textbf{Улучшение CI/CD процессов:}
Для обеспечения быстрой и надежной выкладки новых функциональностей из тестовой среды в продуктовую необходимо улучшить процессы CI/CD. В планах:
\begin{itemize}
    \item Автоматизировать процессы тестирования и развертывания новых версий микросервисов.
    \item Оптимизировать pipeline для сокращения времени развертывания и повышения стабильности релизов.
\end{itemize}

Реализация этих улучшений позволит нам создать более стабильное, функциональное и удобное приложение, которое будет отвечать потребностям пользователей и соответствовать современным стандартам качества.

